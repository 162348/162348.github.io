%\PassOptionsToPackage{top=15truemm,bottom=15truemm,left=10truemm,right=10truemm}{geometry}
\usepackage[top=15truemm,bottom=15truemm,left=10truemm,right=10truemm]{geometry}
\usepackage{picture}
\usepackage{fontawesome5}
\definecolor{minty}{HTML}{80c4ac}\definecolor{花緑青}{cmyk}{1,0.07,0.10,0.10}\definecolor{ISMBlue}{HTML}{2F579C}

% \titleformat{\subsection}[block]
% {}{}{0pt}
% {
%     \colorbox{minty}{\begin{picture}(0,10)\end{picture}}
%     \hspace{0pt}
%     \normalfont \Large\bfseries
%     \hspace{-4pt}
% }
% [
% \begin{picture}(100,0)
%     \put(3,18){\color{minty}\line(1,0){300}}
% \end{picture}
% \\
% \vspace{-30pt}
% ]

% \titlespacing{\subsection}{0pc}{3.5ex plus .1ex minus .2ex}{1.5ex minus .1ex}

% \renewcommand{\labelitemi}{\textcolor{minty}{\faCheckCircle}} %https://mirrors.ibiblio.org/CTAN/fonts/fontawesome/doc/fontawesome.pdf
% \faPaperclip が文献の列挙に良いかも.

\usepackage{comment}
\usepackage{mathtools} %内部でamsmathを呼び出すことに注意.
\usepackage{amsfonts} %mathfrak, mathcal, mathbbなど.
\usepackage{amsthm} %定理環境.
\usepackage{amssymb} %AMSFontsを使うためのパッケージ.
\usepackage{ascmac} %screen, itembox, shadebox環境.全てLATEX2εの標準機能の範囲で作られたもの.
\def\objectstyle{\displaystyle}
% \usepackage{xeCJK}  % ダウンロード長すぎる
% \setCJKmainfont{UDEVGothicNF-Regular.ttf}  % UDEVGothicNF-Regular.ttf が使いたい
% \usepackage{luatexja-fontspec}
% \setmainfont{みかちゃん.ttf}  % 英語だけ動いた.何?
\usepackage{enumerate} %enumerate環境を凝らせる.
\renewcommand{\labelenumi}{(\arabic{enumi})} %(1),(2),...がデフォルトであって欲しい
\renewcommand{\labelenumii}{(\roman{enumii})}
\renewcommand{\labelenumiii}{(\alph{enumiii})}
\usepackage{luatexja}
\usepackage{footnote}