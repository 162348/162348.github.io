%%% Layout
  %\PassOptionsToPackage{top=15truemm,bottom=15truemm,left=10truemm,right=10truemm}{geometry}
\usepackage[top=15truemm,bottom=15truemm,left=10truemm,right=10truemm]{geometry}
% Floats
\usepackage{picture}\usepackage{footnote}
% Decorations
\usepackage{fontawesome5}
\definecolor{minty}{HTML}{80c4ac}\definecolor{ParisGreen}{cmyk}{1,0.07,0.10,0.10}\definecolor{ISMBlue}{HTML}{2F579C}
% Links
\usepackage{url}\PassOptionsToPackage{colorlinks,linkcolor=minty,urlcolor=minty,citecolor=minty}{hyperref}

%%% Language & Mathematics
% Mathematics
\usepackage{mathtools} %内部でamsmathを呼び出すことに注意.
\usepackage{amsfonts} %mathfrak, mathcal, mathbbなど.
\usepackage{amsthm} %定理環境.
\usepackage{amssymb} %AMSFontsを使うためのパッケージ.
\usepackage{ascmac} %screen, itembox, shadebox環境.全てLATEX2εの標準機能の範囲で作られたもの.
\everymath{\displaystyle}
% Fonts
  % \usepackage[math]{anttor}\DeclareMathAlphabet{\mathcal}{OMS}{cmsy}{m}{n}  % LaTeX Error: Math alphabet identifier \mathrm  is undefined in math version `antt'.
% Japanese
\usepackage{luatexja}
% \usepackage{xeCJK}  % ダウンロード長すぎる
% \setCJKmainfont{UDEVGothicNF-Regular.ttf}  % UDEVGothicNF-Regular.ttf が使いたい
% \usepackage{luatexja-fontspec}
% \setmainfont{みかちゃん.ttf}  % 英語だけ動いた.何?


%%% Environments
\usepackage{enumerate} %enumerate環境を凝らせる.
\renewcommand{\labelenumi}{(\arabic{enumi})} %(1),(2),...がデフォルトであって欲しい
\renewcommand{\labelenumii}{(\roman{enumii})}
\renewcommand{\labelenumiii}{(\alph{enumiii})}
% Bibliography
% \usepackage{etoolbox}
% \AtBeginEnvironment{CSLReferences}{% thebibliography 環境の前にフックを追加
%   \setlength{\itemsep}{1pt} % \bibitem 間のスペースを0ptに設定
%   \setlength{\parskip}{1pt} % 段落間のスペースを0ptに設定
% }
% Theorems
\usepackage{amsthm}\newtheoremstyle{StatementsWithUnderline}% ?name?
{3pt}% ?Space above? 1
{3pt}% ?Space below? 1
{}% ?Body font?
{}% ?Indent amount? 2
{\bfseries}% ?Theorem head font?
{\textbf{.}}% ?Punctuation after theorem head?
{.5em}% ?Space after theorem head? 3
{\textbf{\underline{\textup{#1~\thetheorem{}}}}\;\thmnote{(#3)}}% ?Theorem head spec (can be left empty, meaning ‘normal’)?
\definecolor{darkolivegreen}{rgb}{0.33, 0.42, 0.18}\newenvironment{Proof}[1][\bf\underline{[証明]}]{\proof[#1]\color{darkolivegreen}}{\endproof}
\theoremstyle{StatementsWithUnderline}\newtheorem{theorem}{定理}[section]\newtheorem{definition}[theorem]{定義}\newtheorem{corollary}[theorem]{系}\newtheorem{proposition}[theorem]{命題}\newtheorem{lemma}[theorem]{補題}\newtheorem{example}[theorem]{例}
\theoremstyle{definition}\newtheorem{notation}[theorem]{記法}\newtheorem{algorithm}[theorem]{算譜}\newtheorem{remarks}[theorem]{要諦}\newtheorem{remark}[theorem]{注}
\usepackage{etoolbox}\AtEndEnvironment{example}{\hfill\ensuremath{\Box}}\renewcommand{\thefootnote}{\dag\arabic{footnote}}\renewcommand{\thempfootnote}{\dag\arabic{mpfootnote}}
% 行間調整
\renewcommand{\baselinestretch}{1.5}  % なぜか行間がものすごく詰まってしまうので.


% \titleformat{\subsection}[block]
% {}{}{0pt}
% {
%     \colorbox{minty}{\begin{picture}(0,10)\end{picture}}
%     \hspace{0pt}
%     \normalfont \Large\bfseries
%     \hspace{-4pt}
% }
% [
% \begin{picture}(100,0)
%     \put(3,18){\color{minty}\line(1,0){300}}
% \end{picture}
% \\
% \vspace{-30pt}
% ]

% \titlespacing{\subsection}{0pc}{3.5ex plus .1ex minus .2ex}{1.5ex minus .1ex}

% \renewcommand{\labelitemi}{\textcolor{minty}{\faCheckCircle}} %https://mirrors.ibiblio.org/CTAN/fonts/fontawesome/doc/fontawesome.pdf
% \faPaperclip が文献の列挙に良いかも.



