\PassOptionsToPackage{top=15truemm,bottom=15truemm,left=10truemm,right=10truemm}{geometry}
\usepackage{titlesec}
\usepackage{picture}
\usepackage{fontawesome5}
\definecolor{minty}{HTML}{80c4ac}

\titleformat{\subsection}[block]
{}{}{0pt}
{
    \colorbox{minty}{\begin{picture}(0,10)\end{picture}}
    \hspace{0pt}
    \normalfont \Large\bfseries
    \hspace{-4pt}
}
[
\begin{picture}(100,0)
    \put(3,18){\color{minty}\line(1,0){300}}
\end{picture}
\\
\vspace{-30pt}
]

\titlespacing{\subsection}{0pc}{3.5ex plus .1ex minus .2ex}{1.5ex minus .1ex}

\renewcommand{\labelitemi}{\textcolor{minty}{\faCheckCircle}} %https://mirrors.ibiblio.org/CTAN/fonts/fontawesome/doc/fontawesome.pdf
% \faPaperclip が文献の列挙に良いかも.

\usepackage{comment}
\usepackage{mathtools} %内部でamsmathを呼び出すことに注意.
\usepackage{amsfonts} %mathfrak, mathcal, mathbbなど.
\usepackage{amsthm} %定理環境.
\usepackage{amssymb} %AMSFontsを使うためのパッケージ.
\usepackage{ascmac} %screen, itembox, shadebox環境.全てLATEX2εの標準機能の範囲で作られたもの.
\usepackage{wrapfig} %図の周りに文字をwrapさせることができる.詳細な制御ができる.
\usepackage{float} %float option [H]を使える
\setcounter{tocdepth}{2} %目次に表示される深さ.2はsubsectionまで

\usepackage{haranoaji}

\def\objectstyle{\displaystyle}