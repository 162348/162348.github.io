% Options for packages loaded elsewhere
\PassOptionsToPackage{unicode}{hyperref}
\PassOptionsToPackage{hyphens}{url}
\PassOptionsToPackage{dvipsnames,svgnames,x11names}{xcolor}
%
\documentclass[
]{article}

\usepackage{amsmath,amssymb}
\usepackage{iftex}
\ifPDFTeX
  \usepackage[T1]{fontenc}
  \usepackage[utf8]{inputenc}
  \usepackage{textcomp} % provide euro and other symbols
\else % if luatex or xetex
  \usepackage{unicode-math}
  \defaultfontfeatures{Scale=MatchLowercase}
  \defaultfontfeatures[\rmfamily]{Ligatures=TeX,Scale=1}
\fi
\usepackage[sfmath]{kpfonts}
\ifPDFTeX\else  
    % xetex/luatex font selection
\fi
% Use upquote if available, for straight quotes in verbatim environments
\IfFileExists{upquote.sty}{\usepackage{upquote}}{}
\IfFileExists{microtype.sty}{% use microtype if available
  \usepackage[]{microtype}
  \UseMicrotypeSet[protrusion]{basicmath} % disable protrusion for tt fonts
}{}
\makeatletter
\@ifundefined{KOMAClassName}{% if non-KOMA class
  \IfFileExists{parskip.sty}{%
    \usepackage{parskip}
  }{% else
    \setlength{\parindent}{0pt}
    \setlength{\parskip}{6pt plus 2pt minus 1pt}}
}{% if KOMA class
  \KOMAoptions{parskip=half}}
\makeatother
\usepackage{xcolor}
\setlength{\emergencystretch}{3em} % prevent overfull lines
\setcounter{secnumdepth}{5}


\providecommand{\tightlist}{%
  \setlength{\itemsep}{0pt}\setlength{\parskip}{0pt}}\usepackage{longtable,booktabs,array}
\usepackage{calc} % for calculating minipage widths
% Correct order of tables after \paragraph or \subparagraph
\usepackage{etoolbox}
\makeatletter
\patchcmd\longtable{\par}{\if@noskipsec\mbox{}\fi\par}{}{}
\makeatother
% Allow footnotes in longtable head/foot
\IfFileExists{footnotehyper.sty}{\usepackage{footnotehyper}}{\usepackage{footnote}}
\makesavenoteenv{longtable}
\usepackage{graphicx}
\makeatletter
\newsavebox\pandoc@box
\newcommand*\pandocbounded[1]{% scales image to fit in text height/width
  \sbox\pandoc@box{#1}%
  \Gscale@div\@tempa{\textheight}{\dimexpr\ht\pandoc@box+\dp\pandoc@box\relax}%
  \Gscale@div\@tempb{\linewidth}{\wd\pandoc@box}%
  \ifdim\@tempb\p@<\@tempa\p@\let\@tempa\@tempb\fi% select the smaller of both
  \ifdim\@tempa\p@<\p@\scalebox{\@tempa}{\usebox\pandoc@box}%
  \else\usebox{\pandoc@box}%
  \fi%
}
% Set default figure placement to htbp
\def\fps@figure{htbp}
\makeatother
% definitions for citeproc citations
\NewDocumentCommand\citeproctext{}{}
\NewDocumentCommand\citeproc{mm}{%
  \begingroup\def\citeproctext{#2}\cite{#1}\endgroup}
\makeatletter
 % allow citations to break across lines
 \let\@cite@ofmt\@firstofone
 % avoid brackets around text for \cite:
 \def\@biblabel#1{}
 \def\@cite#1#2{{#1\if@tempswa , #2\fi}}
\makeatother
\newlength{\cslhangindent}
\setlength{\cslhangindent}{1.5em}
\newlength{\csllabelwidth}
\setlength{\csllabelwidth}{3em}
\newenvironment{CSLReferences}[2] % #1 hanging-indent, #2 entry-spacing
 {\begin{list}{}{%
  \setlength{\itemindent}{0pt}
  \setlength{\leftmargin}{0pt}
  \setlength{\parsep}{0pt}
  % turn on hanging indent if param 1 is 1
  \ifodd #1
   \setlength{\leftmargin}{\cslhangindent}
   \setlength{\itemindent}{-1\cslhangindent}
  \fi
  % set entry spacing
  \setlength{\itemsep}{#2\baselineskip}}}
 {\end{list}}
\usepackage{calc}
\newcommand{\CSLBlock}[1]{\hfill\break\parbox[t]{\linewidth}{\strut\ignorespaces#1\strut}}
\newcommand{\CSLLeftMargin}[1]{\parbox[t]{\csllabelwidth}{\strut#1\strut}}
\newcommand{\CSLRightInline}[1]{\parbox[t]{\linewidth - \csllabelwidth}{\strut#1\strut}}
\newcommand{\CSLIndent}[1]{\hspace{\cslhangindent}#1}

\makeatletter
\@ifpackageloaded{tcolorbox}{}{\usepackage[skins,breakable]{tcolorbox}}
\@ifpackageloaded{fontawesome5}{}{\usepackage{fontawesome5}}
\definecolor{quarto-callout-color}{HTML}{909090}
\definecolor{quarto-callout-note-color}{HTML}{0758E5}
\definecolor{quarto-callout-important-color}{HTML}{CC1914}
\definecolor{quarto-callout-warning-color}{HTML}{EB9113}
\definecolor{quarto-callout-tip-color}{HTML}{00A047}
\definecolor{quarto-callout-caution-color}{HTML}{FC5300}
\definecolor{quarto-callout-color-frame}{HTML}{acacac}
\definecolor{quarto-callout-note-color-frame}{HTML}{4582ec}
\definecolor{quarto-callout-important-color-frame}{HTML}{d9534f}
\definecolor{quarto-callout-warning-color-frame}{HTML}{f0ad4e}
\definecolor{quarto-callout-tip-color-frame}{HTML}{02b875}
\definecolor{quarto-callout-caution-color-frame}{HTML}{fd7e14}
\makeatother
\makeatletter
\@ifpackageloaded{caption}{}{\usepackage{caption}}
\AtBeginDocument{%
\ifdefined\contentsname
  \renewcommand*\contentsname{Table of contents}
\else
  \newcommand\contentsname{Table of contents}
\fi
\ifdefined\listfigurename
  \renewcommand*\listfigurename{List of Figures}
\else
  \newcommand\listfigurename{List of Figures}
\fi
\ifdefined\listtablename
  \renewcommand*\listtablename{List of Tables}
\else
  \newcommand\listtablename{List of Tables}
\fi
\ifdefined\figurename
  \renewcommand*\figurename{図}
\else
  \newcommand\figurename{図}
\fi
\ifdefined\tablename
  \renewcommand*\tablename{Table}
\else
  \newcommand\tablename{Table}
\fi
}
\@ifpackageloaded{float}{}{\usepackage{float}}
\floatstyle{ruled}
\@ifundefined{c@chapter}{\newfloat{codelisting}{h}{lop}}{\newfloat{codelisting}{h}{lop}[chapter]}
\floatname{codelisting}{Listing}
\newcommand*\listoflistings{\listof{codelisting}{List of Listings}}
\makeatother
\makeatletter
\makeatother
\makeatletter
\@ifpackageloaded{caption}{}{\usepackage{caption}}
\@ifpackageloaded{subcaption}{}{\usepackage{subcaption}}
\makeatother

\usepackage{bookmark}

\IfFileExists{xurl.sty}{\usepackage{xurl}}{} % add URL line breaks if available
\urlstyle{same} % disable monospaced font for URLs
\hypersetup{
  pdftitle={Langevin Dynamics の多項式エルゴード性},
  pdfauthor={司馬博文},
  colorlinks=true,
  linkcolor={minty},
  filecolor={minty},
  citecolor={minty},
  urlcolor={minty},
  pdfcreator={LaTeX via pandoc}}

%\PassOptionsToPackage{top=15truemm,bottom=15truemm,left=10truemm,right=10truemm}{geometry}
\usepackage[top=15truemm,bottom=15truemm,left=10truemm,right=10truemm]{geometry}
\usepackage{picture}
\usepackage{fontawesome5}
\definecolor{minty}{HTML}{80c4ac}\definecolor{ParisGreen}{cmyk}{1,0.07,0.10,0.10}\definecolor{ISMBlue}{HTML}{2F579C}

% \titleformat{\subsection}[block]
% {}{}{0pt}
% {
%     \colorbox{minty}{\begin{picture}(0,10)\end{picture}}
%     \hspace{0pt}
%     \normalfont \Large\bfseries
%     \hspace{-4pt}
% }
% [
% \begin{picture}(100,0)
%     \put(3,18){\color{minty}\line(1,0){300}}
% \end{picture}
% \\
% \vspace{-30pt}
% ]

% \titlespacing{\subsection}{0pc}{3.5ex plus .1ex minus .2ex}{1.5ex minus .1ex}

% \renewcommand{\labelitemi}{\textcolor{minty}{\faCheckCircle}} %https://mirrors.ibiblio.org/CTAN/fonts/fontawesome/doc/fontawesome.pdf
% \faPaperclip が文献の列挙に良いかも.

\usepackage{comment}
\usepackage{mathtools} %内部でamsmathを呼び出すことに注意.
\usepackage{amsfonts} %mathfrak, mathcal, mathbbなど.
\usepackage{amsthm} %定理環境.
\usepackage{amssymb} %AMSFontsを使うためのパッケージ.
\usepackage{ascmac} %screen, itembox, shadebox環境.全てLATEX2εの標準機能の範囲で作られたもの.
\def\objectstyle{\displaystyle}
% \usepackage{xeCJK}  % ダウンロード長すぎる
% \setCJKmainfont{UDEVGothicNF-Regular.ttf}  % UDEVGothicNF-Regular.ttf が使いたい
% \usepackage{luatexja-fontspec}
% \setmainfont{みかちゃん.ttf}  % 英語だけ動いた.何?
\usepackage{enumerate} %enumerate環境を凝らせる.
\renewcommand{\labelenumi}{(\arabic{enumi})} %(1),(2),...がデフォルトであって欲しい
\renewcommand{\labelenumii}{(\roman{enumii})}
\renewcommand{\labelenumiii}{(\alph{enumiii})}
\usepackage{luatexja}
\usepackage{footnote}
\title{Langevin Dynamics の多項式エルゴード性}
\usepackage{etoolbox}
\makeatletter
\providecommand{\subtitle}[1]{% add subtitle to \maketitle
  \apptocmd{\@title}{\par {\large #1 \par}}{}{}
}
\makeatother
\subtitle{Ergodic Lower Bounds}
\author{司馬博文}
\date{7/05/2024}

\begin{document}
\maketitle
\begin{abstract}
目標分布の裾が重ければ重いほど,Langevin
拡散過程の収束は遅くなる.本記事ではその様子を,平衡分布との全変動距離について,定量的に評価する.
\end{abstract}


\(\mathbb{R}^n\) 上の Langevin 拡散を考える:
\begin{equation}\phantomsection\label{eq-Langevin}{
dX_t=-\nabla V(X_t)\,dt+\sqrt{2\beta^{-1}}\,dB_t,\qquad X_0=x.
}\end{equation} ただし, \[
V(x)=O(\lvert x\rvert^{2k})\qquad(\lvert x\rvert\to\infty)
\] の仮定をおく.\(k\ge1/2\) の場合,指数エルゴード的であるが,\(k<1/2\)
の場合はそうではない.

\(k\in(0,1/2)\) の設定で,次の ergodic lower bound を示したい:
\begin{equation}\phantomsection\label{eq-ergodic-lower-bound}{
C_1\exp\left(c_1V(x)-c_2t^{\frac{k}{1-k}}\right)\le\|P_t(x,-)-\mu_*\|_\mathrm{TV}
}\end{equation} \[
\mu_*(dx)\,\propto\,e^{-\beta V(x)}dx
\] この lower bound から,\(k\in(0,1/2)\) の場合,Langevin 過程 \(X\)
が指数エルゴード的たり得ないことが従う.

式 (\ref{eq-ergodic-lower-bound})
を示すためには,\(G(x):=e^{\kappa V(x)}\;(\kappa\in\mathbb{R})\)
に対して, \[
\operatorname{E}_x[G(X_t)]\le g(x,t)
\] を満たす関数 \(g\) を見つける必要がある (Hairer, 2021, pp. 34--35).

これは次の3ステップを辿る

\begin{enumerate}
\def\labelenumi{\arabic{enumi}.}
\tightlist
\item
  そもそも \(\operatorname{E}_x[G(X_t)]<\infty\) であることの証明(第
  \ref{sec-integrability} 節).
\item
  \(G\) に関するドリフト条件 \(P_t\widehat{L}G\le C\varphi\circ G\)
  から,\(\operatorname{E}_x[G(X_t)]\) の \(t\)
  に関する微分不等式を導く(第 \ref{sec-extended-generator} 節).
\item
  微分不等式から,Gronwall の補題より,結論を得る(第 \ref{sec-Gronwall}
  節).
\end{enumerate}

\section{\texorpdfstring{\(G=e^{\kappa V}\)
の可積分性について}{G=e\^{}\{\textbackslash kappa V\} の可積分性について}}\label{sec-integrability}

\begin{tcolorbox}[enhanced jigsaw, opacityback=0, arc=.35mm, left=2mm, rightrule=.15mm, breakable, colframe=quarto-callout-tip-color-frame, bottomrule=.15mm, colback=white, toprule=.15mm, leftrule=.75mm]

次元 \(n=1\) で考えてみる.

\(V(x)=\frac{x^2}{2}\) とした場合,\(X\) は OU 過程になり,

\[
\operatorname{E}_x[G(X_t)]<\infty\quad\Leftrightarrow\quad t<-\frac{1}{2}\log\left(1-\frac{\beta}{\kappa}\right).
\]

\(V(x)=\log x\) とした場合,\(X\) は Bessel 過程になり, \[
\operatorname{E}_x[G(X_t)]<\infty\qquad(\forall_{t>0}).
\]

\(k\in(0,1/2)\) の場合,\(\nabla V\)
が有界であることに注目すれば,Bessel 過程の場合と同様に \[
\operatorname{E}_x[G(X_t)]<\infty\qquad(\forall_{t>0}).
\]

\end{tcolorbox}

\subsection{はじめに}\label{ux306fux3058ux3081ux306b}

Markov 過程 \(X\) に関するドリフト条件 \[
\widehat{L}V\le-C\varphi\circ V\qquad\mathrm{on}\;\mathbb{R}^n\setminus K
\] からは \(V:E\to\mathbb{R}_+\) の可積分性が出る: \[
\operatorname{E}_x[V(X_t)]<\infty\qquad t\ge0.
\]

\begin{tcolorbox}[enhanced jigsaw, opacitybacktitle=0.6, left=2mm, toptitle=1mm, breakable, toprule=.15mm, colbacktitle=quarto-callout-note-color!10!white, leftrule=.75mm, opacityback=0, coltitle=black, rightrule=.15mm, bottomrule=.15mm, title={証明}, colframe=quarto-callout-note-color-frame, bottomtitle=1mm, colback=white, arc=.35mm, titlerule=0mm]

上のドリフト条件を,(Hairer, 2021) の最も弱い意味で解釈すると \[
M_t:=V(X_t)+C\int^t_0\varphi\circ V(X_s)\,ds
\] が任意の \(x\in E\) に関して
\(\operatorname{P}_x\)-局所優マルチンゲールである,ということになる.

これだけの仮定でも,\(V\) が下に有界であるために \(M_t\)
も下に有界であり,下に有界な局所優マルチンゲールは(真の)優マルチンゲールであることから,
\[
\operatorname{E}_x\left[V(X_t)+C\int^t_0\varphi\circ V(X_s)\,ds\right]\le V(x).
\]

加えて左辺が下に有界であることから,\(\operatorname{E}_x[V(X_t)]<\infty\)
でないと矛盾が起こる.

\end{tcolorbox}

しかし,lower bound を得たい場合,
\begin{equation}\phantomsection\label{eq-drift-condition}{
\widehat{L}V\le C\varphi\circ G\qquad\mathrm{on}\;\mathbb{R}^n
}\end{equation} という情報のみから, \[
\operatorname{E}_x[G(X_t)]\le g(x,t)\;(<\infty)
\]
という評価を得る必要が出てくる.この場合,\(\operatorname{E}_x[G(X_t)]<\infty\)
は非自明で,ケースバイケースの議論がである.

\subsection{OU 過程の場合}\label{ou-ux904eux7a0bux306eux5834ux5408}

An overdamped Langevin dynamics on \(\mathbb{R}\) is defined as the
solution to the following SDE: \[
dX_t=-\nabla V(X_t)\,dt+\sqrt{2\beta^{-1}}\,dB_t,\qquad X_0=x_0.
\]

If \(V(x)=\frac{x^2}{2}\), \(X\) becomes an Ornstein-Uhlenbeck process.
Transforming via \(f(t,x)=xe^t\) and using Itô's formula, we get \[
X_t=x_0e^{-t}+\sqrt{2\beta^{-1}}\int^t_0e^{-(t-s)}\,dB_s.
\] Hence, \(X\) is a Gaussian process with
\(X_t\sim\mathrm{N}\left(x_0e^{-t},\beta^{-1}(1-e^{-2t})\right)\).

In this case, expectation with respect to
\(G(y)=e^{\kappa V(y)}=e^{\frac{\kappa y^2}{2}}\;(\kappa\in\mathbb{R})\)
is given by

\begin{align*}
    \operatorname{E}_x[G(X_t)]&=\int_{\mathbb{R}} G(y)\frac{1}{\sqrt{2\pi\beta^{-1}(1-e^{-2t})}}\exp\left(-\frac{(y-xe^{-t})^2}{2\beta^{-1}(1-e^{-2t})}\right)\,dy\\
    &=\frac{1}{\sqrt{2\pi\beta^{-1}(1-e^{-2t})}}\int_{\mathbb{R}}\exp\left(\frac{\kappa\beta^{-1}(1-e^{-2t})y^2-(y-xe^{-t})^2}{2\beta^{-1}(1-e^{-2t})}\right)\,dy.
\end{align*}

Taking a closer look at the numerator inside \(\exp\),

\begin{align*}
    &\qquad\kappa\beta^{-1}(1-e^{-2t})y^2-(y-xe^{-t})^2\\
    &=y^2\biggr(\kappa\beta^{-1}(1-e^{-2t})-1\biggl)-2xe^{-t}y+x^2e^{-2t}.
\end{align*}

Therefore, we conclude \[
\operatorname{E}_x[G(X_t)]<\infty\quad\Leftrightarrow\quad\kappa\beta^{-1}(1-e^{-2t})<1.
\] In other words, \(P_tG(x)\) is finite as long as \[
t<-\frac{1}{2}\log\left(1-\frac{\beta}{\kappa}\right).
\]

\subsection{Bessel
過程の場合}\label{bessel-ux904eux7a0bux306eux5834ux5408}

\(V=a\log x\) ととると,\(V'(x)=\frac{a}{x}\) であるから,これに関する
Langevin 動力学は,\(\beta=1\) のとき, \[
dX_t=-\frac{a}{X_t}\,dt+dB_t
\] と,母数 \(a\) を持つ Bessel 過程になる.ただし,\(0\) への到着時刻
\(T_0\) で止めたもの \(X^{T_0}\) を考える.

\begin{tcolorbox}[enhanced jigsaw, opacitybacktitle=0.6, left=2mm, toptitle=1mm, breakable, toprule=.15mm, colbacktitle=quarto-callout-tip-color!10!white, leftrule=.75mm, opacityback=0, coltitle=black, rightrule=.15mm, bottomrule=.15mm, title={{[}@Lawler2019 p.10 命題2.5{]}}, colframe=quarto-callout-tip-color-frame, bottomtitle=1mm, colback=white, arc=.35mm, titlerule=0mm]

母数 \(a\) を持つ Bessel 過程 \(X^{T_0}\) の密度を \(q_t(x,y;a)\)
で表す.このとき, \[
q_t(x,y;1-a)=\left(\frac{y}{x}\right)^{1-2a}q_t(x,y;a)
\] \[
q_t(x,y;a)=q_t(y,x;a)\left(\frac{y}{x}\right)^{2a}
\] \[
q_{r^2t}(rx,ry;a)=\frac{1}{r}q_t(x,y;a)
\]

加えて \(a\ge\frac{1}{2}\) でもあるとき, \[
q_1(x,y;a)=y^{2a}\exp\left(-\frac{x^2+y^2}{2}\right)h_a(xy),
\] \[
h_a(x)\sim\frac{1}{\sqrt{2\pi}}x^{-a}e^x\qquad(\lvert x\rvert\to\infty)
\]

\end{tcolorbox}

この結果は (Lawler, 2019, p. 59) をみる限り,修正 Bessel 関数と,Bessel
過程の Fokker-Planck 方程式との考察によって証明されている.

\[
G(y)=e^{\kappa V(y)}=e^{a\kappa\log(y)}=y^{a\kappa}
\] であるから,密度 \(q_t(x,y;a)\) に対してはどうやっても可積分である.

\subsection{\texorpdfstring{\(k<1/2\)
の場合の尾部確率}{k\textless1/2 の場合の尾部確率}}\label{k12-ux306eux5834ux5408ux306eux5c3eux90e8ux78baux7387}

\(k<1/2\) で最も大きく変わる点は, \[
\nabla V(x)=O(\lvert x\rvert^{2k-1})\qquad(\lvert x\rvert\to\infty)
\] であるために,\(\nabla V\) が \(\mathbb{R}^n\)
上で有界になることである.

このため,一般に SDE \[
dZ_t=b(Z_t)\,dt+\sigma(X_t)\,dB_t
\] の密度が,任意の \(T>0\) に対して,ある \(A_T,a_T>a\) と
\(y\in\mathbb{R}\) が存在して \[
\frac{1}{A_T\sqrt{2\pi t}}e^{-\frac{a_T\lvert y-x\rvert^2}{2t}}\le p_t(x,y)\le\frac{A_T}{\sqrt{2\pi t}}e^{-\frac{\lvert y-x\rvert^2}{2a_Tt}}
\] \[
t\in(0,T]
\] が成り立つことが使える.\footnote{(Kohatsu-Higa et al., 2022)
  で最初に知った.特に (Kohatsu-Higa, 2003)
  は詳しく扱っており,上からの評価は Malliavin 解析から得られる
  (Taniguchi, 1985).同様にして熱方程式の基本解としても捉えられるが.}

これによれば, \[
G(x)=e^{\kappa V(x)}=O(e^{\kappa\lvert x\rvert^{2k}})\quad(\lvert x\rvert\to\infty)
\] に対して \(p_t\) の尾部が勝つため,\(P_tG(x)<\infty\) である.

\subsection{\texorpdfstring{\(k<1/2\) の場合の \(G\)
の可積分性}{k\textless1/2 の場合の G の可積分性}}\label{k12-ux306eux5834ux5408ux306e-g-ux306eux53efux7a4dux5206ux6027}

\(k<1/2\) の場合,式 (\ref{eq-Langevin})
のドリフト係数が有界になる.このことから,\(G\) の可積分性が,\(X_t\)
の密度の考察に依らず次のように導ける.

\[
M:=\max_{x\in\mathbb{R}^n}\nabla V(x)
\] と定める.\(V(x)=O(\lvert x\rvert^{2k})\;(\lvert x\rvert\to\infty)\)
より,ある \(C>0\) が存在して, \[
V(x)\le C\lvert x\rvert^{2k}\qquad\mathrm{on}\;\mathbb{R}^n.
\] \begin{align*}
    \lvert X_t\rvert&\le\int^t_0\lvert\nabla V(X_t)\rvert\,dt+\sqrt{2\beta^{-1}}\lvert B_t\rvert\\
    &\le Mt+\sqrt{2\beta^{-1}}\lvert B_t\rvert
\end{align*} より, \begin{align*}
    \operatorname{E}_x[\lvert G(X_t)\rvert]&\le\operatorname{E}_x\left[e^{\kappa V(\lvert X_t\rvert)}\right]\\
    &\le\operatorname{E}_x\left[\exp\biggr(\kappa V(M_t+\sqrt{2\beta^{-1}\lvert B_t\rvert})\biggl)\right]\\
    &\le e^{\kappa\lvert Mt\rvert^{2k}}\operatorname{E}_x\left[e^{\kappa 2^k\beta^{-k}\lvert B_t\rvert^{2k}}\right]<\infty.
\end{align*}

\section{微分と拡張生成作用素の関係}\label{sec-extended-generator}

\((X_t)\) を \(E=\mathbb{R}^n\) 上の Feller-Dynkin 過程,\((P_t)\)
をその遷移半群,\(\widehat{L}\) をその拡張生成作用素とする.

\begin{tcolorbox}[enhanced jigsaw, opacitybacktitle=0.6, left=2mm, toptitle=1mm, breakable, toprule=.15mm, colbacktitle=quarto-callout-tip-color!10!white, leftrule=.75mm, opacityback=0, coltitle=black, rightrule=.15mm, bottomrule=.15mm, title={命題 \ref{sec-extended-generator}}, colframe=quarto-callout-tip-color-frame, bottomtitle=1mm, colback=white, arc=.35mm, titlerule=0mm]

\(G\in\mathcal{D}(\widehat{L})\) とする.すなわち, \[
t\mapsto M_t:=G(X_t)-\int^t_0\widehat{L}G(X_s)ds
\] は任意の \(x\in E\) について
\(\operatorname{P}_x\)-局所マルチンゲールである.

このとき,さらに \(G\) について次の条件を仮定する:

\begin{enumerate}
\def\labelenumi{\arabic{enumi}.}
\tightlist
\item
  \(\operatorname{E}_x[\lvert G(X_t)\rvert]<\infty\;(x\in E,t\in\mathbb{R}_+)\).すなわち,\(P_tG:E\to\mathbb{R}\)
  が定まる.
\item
  同様に
  \(\operatorname{E}_x[\lvert\widehat{L}(G)(X_t)\rvert]<\infty\;(x\in E,t\in\mathbb{R}_+)\).すなわち,\(\widehat{L}P_tG:E\to\mathbb{R}\)
  も定まる.\footnotemark{}
\item
  \(t\mapsto P_t\widehat{L}G(x)\) は局所有界.
\end{enumerate}

このとき,\(P_tG(x)\) は \(t\) で微分可能であり,次が導ける: \[
\frac{\partial }{\partial t}\operatorname{E}_x[G(X_t)]=\operatorname{E}_x[\widehat{L}G(X_t)].
\]

\end{tcolorbox}

\footnotetext{元々はある正の定数 \(C>0\)
が存在して,\(\widehat{L}G\le CG\).ある凹関数 \(\varphi\) について
\(\widehat{L}G\le\varphi\circ G\)
が成り立つならばこの仮定は満たされることに注意,としていた.}

これは,通常の意味での生成作用素 \(L\) の性質 \[
\frac{\partial }{\partial t}P_tG=P_t(LG)
\] が,可積分性の条件の下で,拡張生成作用素 \(\widehat{L}\)
にも引き継がれると理解できる.

\begin{tcolorbox}[enhanced jigsaw, opacitybacktitle=0.6, left=2mm, toptitle=1mm, breakable, toprule=.15mm, colbacktitle=quarto-callout-note-color!10!white, leftrule=.75mm, opacityback=0, coltitle=black, rightrule=.15mm, bottomrule=.15mm, title={証明}, colframe=quarto-callout-note-color-frame, bottomtitle=1mm, colback=white, arc=.35mm, titlerule=0mm]

仮定より,停止時の列 \(\tau_n\nearrow\infty\;\;\text{a.s.}\)
が存在し,任意の \(n\in\mathbb{N}\) について,\(M^{\tau_n}\)
はマルチンゲールで, \begin{equation}\phantomsection\label{eq-1}{
\operatorname{E}_x\left[G(X_{t\land\tau_n})-\int^{t\land\tau_n}_0\widehat{L}G(X_s)ds\right]=G(x),\qquad  t\ge0,x\in E.
}\end{equation}

{仮定1}より
\(\operatorname{E}_x[\lvert G(X_{t\land\tau_n})\rvert]<\infty\)
であるから, \[
\operatorname{E}_x\left[\left|\int^{t\land\tau_n}_0\widehat{L}G(X_s)ds\right|\right]<\infty.
\] でもある.従って Fubini-Tonelli の定理から \[
\operatorname{E}_x\left[\left|\int^{t\land\tau_n}_0\widehat{L}G(X_s)ds\right|\right]=\int^t_0\operatorname{E}_x\biggl[1_{[0,\tau_n]}(s)\widehat{L}G(X_s)\biggr]\,ds
\] と書き換えられる.

よって,式 (\ref{eq-1}) は \[
\operatorname{E}_x\biggl[G(X_{t\land\tau_n})\biggr]=G(x)+\int^{t}_0\operatorname{E}_x\biggl[1_{[0,\tau_n]}(s)\widehat{L}G(X_s)\biggr]\,ds
\] とも表せる.右辺が \(t\)
について微分可能であるから,左辺も微分可能である: \[
\frac{\partial }{\partial t}\operatorname{E}_x\biggl[G(X_{t\land\tau_n})\biggr]=\operatorname{E}_x\biggl[1_{[0,\tau_n]}(t)\widehat{L}G(X_t)\biggr].
\]

両辺の \(n\to\infty\) に関する極限を取ると,右辺は
\(\lvert\widehat{L}G(X_t)\rvert\) が
\(\operatorname{P}_x\)-可積分であるから({仮定2}),Lebesgue
の優収束定理より, \[
\lim_{n\to\infty}\frac{\partial }{\partial t}\operatorname{E}_x[G(X_{t\land\tau_n})]=\lim_{n\to\infty}\operatorname{E}_x\biggl[1_{[0,\tau_n]}(t)\widehat{L}G(X_t)\biggr]=\operatorname{E}_x[\widehat{L}G(X_t)],\qquad x\in E,t\in(0,\infty).
\]

加えてこの収束は,\(t\in(0,\infty)\)
に関して広義一様に起こる.実際,Hölder の不等式より,\footnotemark{}
\begin{align*}
&\qquad\sup_{t\in[0,T]}\left|\frac{\partial }{\partial t}\operatorname{E}_x[G(X_{t\land\tau_n})]-\operatorname{E}_x[\widehat{L}G(X_t)]\right|\\
&=\sup_{t\in[0,T]}\biggl|\operatorname{E}_x[1_{[0,\tau_n]}(t)\widehat{L}G(X_t)]-\operatorname{E}_x[\widehat{L}G(X_t)]\biggr|\\
&=\sup_{t\in[0,T]}\biggl|\operatorname{E}_x\biggl[(1-1_{[0,\tau_n]}(t))\widehat{L}G(X_t)\biggr]\biggr|\\
&\le\sup_{t\in[0,T]}\operatorname{E}_x\biggl[(1-1_{[0,\tau_n]}(T))\lvert\widehat{L}G(X_t)\rvert\biggr]\\
&\le\|1-1_{[0,\tau_n]}(T)\|_{L^\infty(\Omega)}\sup_{t\in[0,T]}\operatorname{E}_x\left[\lvert\widehat{L}G(X_t)\rvert\right]\xrightarrow{n\to\infty}0.
\end{align*} 最後の不等式にて,{仮定3}による局所有界性 \[
\sup_{t\in[0,T]}\operatorname{E}_x\left[\lvert\widehat{L}G(X_t)\rvert\right]<\infty
\] を用いた.

この導関数の一様収束と,Lebesgue の優収束定理による各点収束 \[
\operatorname{E}_x[G(X_{t\land\tau_n})]\xrightarrow{n\to\infty}\operatorname{E}_x[G(X_t)]
\] を併せると,\(\operatorname{E}_x[G(X_t)]\)
も可微分で,その導関数は極限 \[
\frac{\partial }{\partial t}\operatorname{E}_x[G(X_t)]=\lim_{n\to\infty}\frac{\partial }{\partial t}\operatorname{E}_x[G(X_{t\land\tau_n})]=\operatorname{E}_x[\widehat{L}G(X_t)]
\] として得られることが結論づけられる.

\end{tcolorbox}

\footnotetext{\(\sup_{t\in[0,T]}\widehat{L}G(X_t)\)
は可積分とは限らないため,\(\sup\)
を期待値の中に入れることはできない.Hölder
の不等式により,これを迂回できる.}

\begin{tcolorbox}[enhanced jigsaw, opacitybacktitle=0.6, left=2mm, toptitle=1mm, breakable, toprule=.15mm, colbacktitle=quarto-callout-tip-color!10!white, leftrule=.75mm, opacityback=0, coltitle=black, rightrule=.15mm, bottomrule=.15mm, title={{[}@Rudin-Principles p.152 定理7.17{]}\footnote{{[}@杉浦光夫1980
  p.311{]} 定理13.7系では,\(f_n\) に
  \(C^1\)-級の仮定を置いて,この場合は \(f\) が
  \(C^1\)-級になることを導いている.}}, colframe=quarto-callout-tip-color-frame, bottomtitle=1mm, colback=white, arc=.35mm, titlerule=0mm]

\(f_n:[a,b]\to\mathbb{R}\) を可微分な関数列とし,ある関数
\(f:[a,b]\to\mathbb{R}\) に各点収束するものとする.

仮に,導関数列 \(\{f'_n\}\) が一様位相に関して Cauchy
列ならば,\(f_n\to f\) も一様収束し,加えて \(f\) も可微分で, \[
\lim_{n\to\infty}f'_n(x)=f'(x)
\] が成り立つ.

\end{tcolorbox}

\section{下界の導出}\label{sec-Gronwall}

元来の目的である下界の導出のためには, \[
\operatorname{E}_x[G(X_t)]\le CG(x)\exp\left(ct^{\frac{k}{1-k}}\right)
\] という評価を得る必要がある.Gronwall
の不等式を用いれば,導関数に関する不等式 \[
\frac{\partial }{\partial t}\operatorname{E}_x[G(X_t)]\le\operatorname{E}_x[\widehat{L}G(X_t)]\le C\operatorname{E}_x[\varphi\circ G(X_t)]
\] があれば十分である.この導関数に関する不等式は,命題
\ref{sec-extended-generator} とドリフト条件 (\ref{eq-drift-condition})
\[
\widehat{L}G\le C\varphi\circ G
\] を併せることで, \[
\frac{\partial }{\partial t}\operatorname{E}_x[G(X_t)]=\operatorname{E}_x[\widehat{L}G(X_t)]\le C\operatorname{E}_x[\varphi\circ G(X_t)]
\] より得る.

\section*{参考文献}\label{ux53c2ux8003ux6587ux732e}
\addcontentsline{toc}{section}{参考文献}

\phantomsection\label{refs}
\begin{CSLReferences}{1}{1}
\bibitem[\citeproctext]{ref-Hairer2021-Convergence}
Hairer, M. (2021).
\emph{\href{https://www.hairer.org/notes/Convergence.pdf}{Convergence of
markov processes}}.

\bibitem[\citeproctext]{ref-Kohatsu-Higa2003}
Kohatsu-Higa, A. (2003).
\href{https://doi.org/10.1007/978-3-0348-8069-5_18}{Lower bounds for
densities of uniformly elliptic non-homogeneous diffusions}. In E. Giné,
C. Houdré, and D. Nualart, editors, \emph{Stochastic inequalities and
applications}, pages 323--338. Basel: Birkh{ä}user Basel.

\bibitem[\citeproctext]{ref-Kohatsu-Higa2022}
Kohatsu-Higa, A., Nualart, E., and Tran, N. K. (2022).
\href{https://doi.org/10.1016/j.amc.2021.126814}{Density estimates for
jump diffusion processes}. \emph{Applied Mathematics and Computation},
\emph{420}, 126814.

\bibitem[\citeproctext]{ref-Lawler2019}
Lawler, G. F. (2019).
\emph{\href{https://www.math.uchicago.edu/~lawler/bessel18new.pdf}{Notes
on the bessel process}}.

\bibitem[\citeproctext]{ref-Taniguchi1985}
Taniguchi, S. (1985).
\href{https://projecteuclid.org/journals/osaka-journal-of-mathematics/volume-22/issue-2/Applications-of-Malliavins-calculus-to-time-dependent-systems-of-heat/ojm/1200778261.full}{{Applications
of Malliavin's calculus to time-dependent systems of heat equations}}.
\emph{Osaka Journal of Mathematics}, \emph{22}(2), 307--320.

\end{CSLReferences}




\end{document}
