% Options for packages loaded elsewhere
\PassOptionsToPackage{unicode}{hyperref}
\PassOptionsToPackage{hyphens}{url}
%
\documentclass[
  ignorenonframetext,
]{beamer}
\usepackage{pgfpages}
\setbeamertemplate{caption}[numbered]
\setbeamertemplate{caption label separator}{: }
\setbeamercolor{caption name}{fg=normal text.fg}
\beamertemplatenavigationsymbolsempty
% Prevent slide breaks in the middle of a paragraph
\widowpenalties 1 10000
\raggedbottom
\setbeamertemplate{part page}{
  \centering
  \begin{beamercolorbox}[sep=16pt,center]{part title}
    \usebeamerfont{part title}\insertpart\par
  \end{beamercolorbox}
}
\setbeamertemplate{section page}{
  \centering
  \begin{beamercolorbox}[sep=12pt,center]{part title}
    \usebeamerfont{section title}\insertsection\par
  \end{beamercolorbox}
}
\setbeamertemplate{subsection page}{
  \centering
  \begin{beamercolorbox}[sep=8pt,center]{part title}
    \usebeamerfont{subsection title}\insertsubsection\par
  \end{beamercolorbox}
}
\AtBeginPart{
  \frame{\partpage}
}
\AtBeginSection{
  \ifbibliography
  \else
    \frame{\sectionpage}
  \fi
}
\AtBeginSubsection{
  \frame{\subsectionpage}
}

\usepackage{amsmath,amssymb}
\usepackage{iftex}
\ifPDFTeX
  \usepackage[T1]{fontenc}
  \usepackage[utf8]{inputenc}
  \usepackage{textcomp} % provide euro and other symbols
\else % if luatex or xetex
  \usepackage{unicode-math}
  \defaultfontfeatures{Scale=MatchLowercase}
  \defaultfontfeatures[\rmfamily]{Ligatures=TeX,Scale=1}
\fi
\usepackage{lmodern}
\ifPDFTeX\else  
    % xetex/luatex font selection
\fi
% Use upquote if available, for straight quotes in verbatim environments
\IfFileExists{upquote.sty}{\usepackage{upquote}}{}
\IfFileExists{microtype.sty}{% use microtype if available
  \usepackage[]{microtype}
  \UseMicrotypeSet[protrusion]{basicmath} % disable protrusion for tt fonts
}{}
\makeatletter
\@ifundefined{KOMAClassName}{% if non-KOMA class
  \IfFileExists{parskip.sty}{%
    \usepackage{parskip}
  }{% else
    \setlength{\parindent}{0pt}
    \setlength{\parskip}{6pt plus 2pt minus 1pt}}
}{% if KOMA class
  \KOMAoptions{parskip=half}}
\makeatother
\usepackage{xcolor}
\newif\ifbibliography
\setlength{\emergencystretch}{3em} % prevent overfull lines
\setcounter{secnumdepth}{-\maxdimen} % remove section numbering


\providecommand{\tightlist}{%
  \setlength{\itemsep}{0pt}\setlength{\parskip}{0pt}}\usepackage{longtable,booktabs,array}
\usepackage{calc} % for calculating minipage widths
\usepackage{caption}
% Make caption package work with longtable
\makeatletter
\def\fnum@table{\tablename~\thetable}
\makeatother
\usepackage{graphicx}
\makeatletter
\def\maxwidth{\ifdim\Gin@nat@width>\linewidth\linewidth\else\Gin@nat@width\fi}
\def\maxheight{\ifdim\Gin@nat@height>\textheight\textheight\else\Gin@nat@height\fi}
\makeatother
% Scale images if necessary, so that they will not overflow the page
% margins by default, and it is still possible to overwrite the defaults
% using explicit options in \includegraphics[width, height, ...]{}
\setkeys{Gin}{width=\maxwidth,height=\maxheight,keepaspectratio}
% Set default figure placement to htbp
\makeatletter
\def\fps@figure{htbp}
\makeatother
% definitions for citeproc citations
\NewDocumentCommand\citeproctext{}{}
\NewDocumentCommand\citeproc{mm}{%
  \begingroup\def\citeproctext{#2}\cite{#1}\endgroup}
\makeatletter
 % allow citations to break across lines
 \let\@cite@ofmt\@firstofone
 % avoid brackets around text for \cite:
 \def\@biblabel#1{}
 \def\@cite#1#2{{#1\if@tempswa , #2\fi}}
\makeatother
\newlength{\cslhangindent}
\setlength{\cslhangindent}{1.5em}
\newlength{\csllabelwidth}
\setlength{\csllabelwidth}{3em}
\newenvironment{CSLReferences}[2] % #1 hanging-indent, #2 entry-spacing
 {\begin{list}{}{%
  \setlength{\itemindent}{0pt}
  \setlength{\leftmargin}{0pt}
  \setlength{\parsep}{0pt}
  % turn on hanging indent if param 1 is 1
  \ifodd #1
   \setlength{\leftmargin}{\cslhangindent}
   \setlength{\itemindent}{-1\cslhangindent}
  \fi
  % set entry spacing
  \setlength{\itemsep}{#2\baselineskip}}}
 {\end{list}}
\usepackage{calc}
\newcommand{\CSLBlock}[1]{\hfill\break\parbox[t]{\linewidth}{\strut\ignorespaces#1\strut}}
\newcommand{\CSLLeftMargin}[1]{\parbox[t]{\csllabelwidth}{\strut#1\strut}}
\newcommand{\CSLRightInline}[1]{\parbox[t]{\linewidth - \csllabelwidth}{\strut#1\strut}}
\newcommand{\CSLIndent}[1]{\hspace{\cslhangindent}#1}

\makeatletter
\@ifpackageloaded{tcolorbox}{}{\usepackage[skins,breakable]{tcolorbox}}
\@ifpackageloaded{fontawesome5}{}{\usepackage{fontawesome5}}
\definecolor{quarto-callout-color}{HTML}{909090}
\definecolor{quarto-callout-note-color}{HTML}{0758E5}
\definecolor{quarto-callout-important-color}{HTML}{CC1914}
\definecolor{quarto-callout-warning-color}{HTML}{EB9113}
\definecolor{quarto-callout-tip-color}{HTML}{00A047}
\definecolor{quarto-callout-caution-color}{HTML}{FC5300}
\definecolor{quarto-callout-color-frame}{HTML}{acacac}
\definecolor{quarto-callout-note-color-frame}{HTML}{4582ec}
\definecolor{quarto-callout-important-color-frame}{HTML}{d9534f}
\definecolor{quarto-callout-warning-color-frame}{HTML}{f0ad4e}
\definecolor{quarto-callout-tip-color-frame}{HTML}{02b875}
\definecolor{quarto-callout-caution-color-frame}{HTML}{fd7e14}
\makeatother
\makeatletter
\@ifpackageloaded{caption}{}{\usepackage{caption}}
\AtBeginDocument{%
\ifdefined\contentsname
  \renewcommand*\contentsname{Table of contents}
\else
  \newcommand\contentsname{Table of contents}
\fi
\ifdefined\listfigurename
  \renewcommand*\listfigurename{List of Figures}
\else
  \newcommand\listfigurename{List of Figures}
\fi
\ifdefined\listtablename
  \renewcommand*\listtablename{List of Tables}
\else
  \newcommand\listtablename{List of Tables}
\fi
\ifdefined\figurename
  \renewcommand*\figurename{図}
\else
  \newcommand\figurename{図}
\fi
\ifdefined\tablename
  \renewcommand*\tablename{Table}
\else
  \newcommand\tablename{Table}
\fi
}
\@ifpackageloaded{float}{}{\usepackage{float}}
\floatstyle{ruled}
\@ifundefined{c@chapter}{\newfloat{codelisting}{h}{lop}}{\newfloat{codelisting}{h}{lop}[chapter]}
\floatname{codelisting}{Listing}
\newcommand*\listoflistings{\listof{codelisting}{List of Listings}}
\makeatother
\makeatletter
\makeatother
\makeatletter
\@ifpackageloaded{caption}{}{\usepackage{caption}}
\@ifpackageloaded{subcaption}{}{\usepackage{subcaption}}
\makeatother
\ifLuaTeX
  \usepackage{selnolig}  % disable illegal ligatures
\fi
\usepackage{bookmark}

\IfFileExists{xurl.sty}{\usepackage{xurl}}{} % add URL line breaks if available
\urlstyle{same} % disable monospaced font for URLs
\hypersetup{
  pdftitle={ベイズとは何か},
  pdfauthor={司馬博文},
  hidelinks,
  pdfcreator={LaTeX via pandoc}}

\title{ベイズとは何か}
\subtitle{数学による統一的アプローチ}
\author{司馬博文}
\date{4/28/2024}

\begin{document}
\frame{\titlepage}

\begin{frame}
\begin{itemize}
\tightlist
\item
  \href{https://ja.wikipedia.org/wiki/\%E3\%83\%88\%E3\%83\%BC\%E3\%83\%9E\%E3\%82\%B9\%E3\%83\%BB\%E3\%83\%99\%E3\%82\%A4\%E3\%82\%BA}{トーマス・ベイズ
  1701-1706}:イギリスの牧師・数学者
\item
  ベイズの定理:確率論において,条件付き確率の計算手段を与える定理
\item
  ベイズ○○:○○(分野名)におけるベイズの定理の応用

  \begin{itemize}
  \tightlist
  \item
    例:ベイズ統計,ベイズ機械学習,ベイズ推論,\ldots\ldots{}
  \item
    例外:ベイズ計算(ベイズの定理の通りに実際に計算をするための\textbf{計算手法の総称})
  \end{itemize}
\end{itemize}

多くの応用を持つが,原理は同一である.

ベイズ深層学習,ベイズ最適化,\ldots\ldots{}
\end{frame}

\begin{frame}{Who: ベイズとは誰か?}
\phantomsection\label{who-ux30d9ux30a4ux30baux3068ux306fux8ab0ux304b}
\begin{block}{始まりは区間推定の問題であった}
\phantomsection\label{ux59cbux307eux308aux306fux533aux9593ux63a8ux5b9aux306eux554fux984cux3067ux3042ux3063ux305f}
\begin{columns}[T]
\begin{column}{0.5\textwidth}
\begin{tcolorbox}[enhanced jigsaw, breakable, colback=white, title={ベイズが取り組んだ問題(現代語訳)\footnote{{[}@Bayes1763{]}}}, toptitle=1mm, toprule=.15mm, colbacktitle=quarto-callout-tip-color!10!white, coltitle=black, left=2mm, opacitybacktitle=0.6, colframe=quarto-callout-tip-color-frame, titlerule=0mm, bottomrule=.15mm, bottomtitle=1mm, arc=.35mm, leftrule=.75mm, rightrule=.15mm, opacityback=0]

2値の確率変数は \(Y_i\in\{0,1\}\) はある確率 \(\theta\in(0,1)\) で
\(1\) になるとする: \[
Y_i=\begin{cases}
1&\text{確率 }\theta\text{ で}\\
0&\text{残りの確率} 1-\theta\text{ で}
\end{cases}
\] このような確率変数の独立な観測 \(y_1,\cdots,y_n\) から,ある区間
\((a,b)\subset[0,1]\) に \(\theta\)
が入っているという確率を計算するにはどうすれば良いか?

\end{tcolorbox}
\end{column}

\begin{column}{0.5\textwidth}
\begin{itemize}
\tightlist
\item
  決定的特徴:未知のパラメータ \(\theta\) に対する確率分布を考えている.
\item
  与えられている観測のモデル \(p(y|\theta)\) に対して,逆の条件付き確率
  \(p(\theta|y)\) を考えれば良い.
\item
  そのための計算公式として「ベイズの定理」を導いた (Bayes, 1763).
\end{itemize}
\end{column}
\end{columns}
\end{block}
\end{frame}

\begin{frame}{What: ベイズとは何か?}
\phantomsection\label{what-ux30d9ux30a4ux30baux3068ux306fux4f55ux304b}
\begin{block}{ベイズの定理}
\phantomsection\label{ux30d9ux30a4ux30baux306eux5b9aux7406}
\begin{tcolorbox}[enhanced jigsaw, breakable, colback=white, title={ベイズの定理\footnote{{[}@Shiryaev2016 p.272{]} (34) も参照.}}, toptitle=1mm, toprule=.15mm, colbacktitle=quarto-callout-tip-color!10!white, coltitle=black, left=2mm, opacitybacktitle=0.6, colframe=quarto-callout-tip-color-frame, titlerule=0mm, bottomrule=.15mm, bottomtitle=1mm, arc=.35mm, leftrule=.75mm, rightrule=.15mm, opacityback=0]

任意の可積分関数 \(g\),確率変数
\(\Theta\sim\operatorname{P}^\Theta\),部分 \(\sigma\)-代数
\(\mathcal{G}\) について, \[
\operatorname{E}[g(\Theta)|\mathcal{G}](\omega)=\frac{\int_\mathbb{R}g(\theta)p(\omega|\theta)\operatorname{P}^{\Theta}(d\theta)}{\int_\mathbb{R}p(\omega|\theta)\operatorname{P}^\Theta(d\theta)}\;\;\text{a.s.}\,\omega
\]

\end{tcolorbox}

一般には次の形で使う: \[
p(\theta|x)=\frac{p(x|\theta)p(\theta)}{\int_\Theta p(x|\theta)p(\theta)\,d\theta}
\]

\begin{tcolorbox}[enhanced jigsaw, breakable, colback=white, title={証明}, toptitle=1mm, toprule=.15mm, colbacktitle=quarto-callout-note-color!10!white, coltitle=black, left=2mm, opacitybacktitle=0.6, colframe=quarto-callout-note-color-frame, titlerule=0mm, bottomrule=.15mm, bottomtitle=1mm, arc=.35mm, leftrule=.75mm, rightrule=.15mm, opacityback=0]

確率空間を \((\Omega,\mathcal{F},\operatorname{P})\),確率変数
\(\Theta\) は可測関数 \(\Omega\to\mathcal{X}\),可積分関数は
\(g\in\mathcal{L}(\mathcal{X})\) とし,定理の式は確率測度
\(\operatorname{P}\) に関して確率 \(1\)
で成り立つという意味であるとした.

可測空間 \((\Omega,\mathcal{G})\) 上の測度 \(\operatorname{Q}\) を \[
\operatorname{Q}(B):=\int_B g(\theta(\omega))\operatorname{P}(d\omega),\qquad B\in\mathcal{G}
\] と定めると, \[
\operatorname{E}[g(\Theta)|\mathcal{G}]=\frac{d \operatorname{Q}}{d \operatorname{P}}.
\] なお,この定理は暗黙に条件付き期待値 \(\operatorname{P}[B|\Theta]\)
は正則で,\((\Omega,\mathcal{G})\) 上の \(\sigma\)-有限な参照測度
\(\lambda\) に対して次の密度を持つことを仮定した: \[
\operatorname{P}[B|\Theta=\theta]=\int_B p(\omega|\theta)\lambda(d\omega).
\] この下では,Fubini の定理から \[
\begin{align*}
  \operatorname{P}[B]&=\int_\mathbb{R}\operatorname{P}[B|\Theta=\theta]\operatorname{P}^\Theta(d\theta)\\
  &=\int_B\int_\mathbb{R}p(\omega|\theta)\operatorname{P}^\Theta(d\theta)\lambda(d\omega)
\end{align*}
\] \[
\begin{align*}
  \operatorname{Q}[B]&=\operatorname{E}[g(\Theta)\operatorname{E}[1_B|\sigma[\Theta]]]\\
  &=\int_\mathbb{R}g(\theta)\operatorname{P}[B|\Theta=\theta]\operatorname{P}^\Theta(d\theta)\\
  &=\int_B\int_\mathbb{R}g(\theta)p(\omega|\theta)\operatorname{P}^\Theta(d\theta)\lambda(d\omega).
\end{align*}
\] よってあとは \[
\frac{d \operatorname{Q}}{d \operatorname{P}}=\frac{d \operatorname{Q}/d\lambda}{d \operatorname{P}/d\lambda}\;\operatorname{P}\text{-a.s.}
\] を示せば良い.これは (Shiryaev, 2016, p. 273) に譲る.

\end{tcolorbox}
\end{block}

\begin{block}{ベイズ推論のもう一つのピース「事前分布」}
\phantomsection\label{ux30d9ux30a4ux30baux63a8ux8ad6ux306eux3082ux3046ux4e00ux3064ux306eux30d4ux30fcux30b9ux4e8bux524dux5206ux5e03}
\end{block}

\begin{block}{帰納的推論の確率的拡張としてのベイズ推論}
\phantomsection\label{ux5e30ux7d0dux7684ux63a8ux8ad6ux306eux78baux7387ux7684ux62e1ux5f35ux3068ux3057ux3066ux306eux30d9ux30a4ux30baux63a8ux8ad6}
\end{block}

\begin{block}{生物の不確実性の下での推論のモデルとしてのベイズ推論}
\phantomsection\label{ux751fux7269ux306eux4e0dux78baux5b9fux6027ux306eux4e0bux3067ux306eux63a8ux8ad6ux306eux30e2ux30c7ux30ebux3068ux3057ux3066ux306eux30d9ux30a4ux30baux63a8ux8ad6}
\begin{itemize}
\item
  脳の平時の活動は経験的事前分布を表現していると解釈できる (Berkes et
  al., 2011)
\item
  脳の神経回路はベイズ推論(正確には,事後分布からのサンプリング)を行っている可能性がある
  (Terada and Toyoizumi, 2024)
\end{itemize}
\end{block}
\end{frame}

\begin{frame}{How: ベイズはどう使うのか?}
\phantomsection\label{how-ux30d9ux30a4ux30baux306fux3069ux3046ux4f7fux3046ux306eux304b}
\begin{block}{「ベイズ計算」という分野}
\phantomsection\label{ux30d9ux30a4ux30baux8a08ux7b97ux3068ux3044ux3046ux5206ux91ce}
\[
p(\theta|x)=\frac{p(x|\theta)p(\theta)}{\int_\Theta p(x|\theta)p(\theta)\,d\theta}
\]

\begin{itemize}
\item
  ベイズの定理で終わりじゃない.

  →「どう実際に計算するか?」(特に分母の積分が問題)
\item
  ベイズ統計,ベイズ機械学習\ldots\ldots{}
  はすべてベイズの定理を使っている.

  →効率的で汎用的な計算方法を1つ見つければ,多くの応用分野に資する.
\end{itemize}
\end{block}

\begin{block}{「ベイズ計算」の問題意識}
\phantomsection\label{ux30d9ux30a4ux30baux8a08ux7b97ux306eux554fux984cux610fux8b58}
\begin{itemize}
\item
  受験問題で出題される積分問題は,解析的に解ける異例中の異例
\item
  加えて,「解析的に解ける」もののみを扱うのでは,モデリングの幅が狭すぎる
\end{itemize}

どんな関数 \(p(x|\theta),p(\theta)\) に対しても積分 \[
\int_\Theta p(x|\theta)p(\theta)\,d\theta
\] が計算できる方法が欲しい.
\end{block}

\begin{block}{積分はどう計算すれば良いか?}
\phantomsection\label{ux7a4dux5206ux306fux3069ux3046ux8a08ux7b97ux3059ux308cux3070ux826fux3044ux304b}
\begin{itemize}
\item
  数値積分(グリッド法)

  → Riemann 積分の定義を地で行く計算法

  → 3次元以上でもう現実的には計算量が爆発する
\item
  モンテカルロ積分法

  → 確定的なグリッドを用いるのではなく,乱数を用いる
\end{itemize}

\begin{quote}
It is evidently impractical to carry out a several hundred-dimensional
integral by the usual numerical methods, so we resort to the Monte Carlo
method. (Metropolis et al., 1953, p. 1088)
\end{quote}
\end{block}
\end{frame}

\begin{frame}{When: ベイズはいつ使えるか?}
\phantomsection\label{when-ux30d9ux30a4ux30baux306fux3044ux3064ux4f7fux3048ux308bux304b}
\end{frame}

\begin{frame}{Why: なぜベイズなのか?}
\phantomsection\label{why-ux306aux305cux30d9ux30a4ux30baux306aux306eux304b}
\end{frame}

\begin{frame}{参考文献}
\phantomsection\label{ux53c2ux8003ux6587ux732e}
\phantomsection\label{refs}
\begin{CSLReferences}{1}{1}
\bibitem[\citeproctext]{ref-Bayes1763}
Bayes, T. (1763). \href{https://www.jstor.org/stable/105741}{An essay
towards solving a problem in the doctrine of chances. By the late rev.
Mr. Bayes, f. R. S. Communicated by mr. Price, in a letter to john
canton, a. M. F. R. s.} \emph{Philosophical Transactions},
\emph{53}(1763), 370--418.

\bibitem[\citeproctext]{ref-Berkes+2011}
Berkes, P., Orbán, G., Lengyel, M., and Fiser, J. (2011).
\href{https://doi.org/10.1126/science.1195870}{Spontaneous cortical
activity reveals hallmarks of an optimal internal model of the
environment}. \emph{Science}, \emph{331}(6013), 83--87.

\bibitem[\citeproctext]{ref-Metropolis+1953}
Metropolis, N., Rosenbluth, A. W., Rosenbluth, M. N., Teller, A. H., and
Teller, E. (1953). \href{https://doi.org/10.1063/1.1699114}{Equation of
state calculations by fast computing machines}. \emph{The Journal of
Chemical Physics}, \emph{21}(6), 1087--1092.

\bibitem[\citeproctext]{ref-Shiryaev2016}
Shiryaev, A. N. (2016).
\emph{\href{https://link.springer.com/book/10.1007/978-0-387-72206-1}{Probability-1}},Vol.
95. Springer New York.

\bibitem[\citeproctext]{ref-Terada-Toyoizumi2024}
Terada, Y., and Toyoizumi, T. (2024).
\href{https://doi.org/10.1073/pnas.2312992121}{Chaotic neural dynamics
facilitate probabilistic computations through sampling}.
\emph{Proceedings of the National Academy of Sciences}, \emph{121}(18),
e2312992121.

\end{CSLReferences}
\end{frame}



\end{document}
