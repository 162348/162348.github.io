%%
%% サンプル
%%

\documentclass[leqno,a4j]{jarticle}
%以下の2行は,式番号を章ごとに1.1,1.2,2.1,2.2,...とつけるための設定
\usepackage{amsmath}
\numberwithin{equation}{section}

\pagestyle{empty}

\begin{document}
\normalsize

\begin{center}

\textbf{\huge{動き出す次世代サンプラー\\区分確定的モンテカルロ}}
\vspace{0.5cm}

%%% 専攻名・学年を以下に入力
%%% 3年制で入学された方も5年一貫制で学年(3~5年)を記入して下さい.
%%% 2023年4月以降に入学された方は「統計科学コース 博士後期課程○年」(5年一貫制による学年(3~5年)を記入)として下さい.
\large{統計科学コース 5年一貫制博士課程2年\\

%%% 発表者氏名
司馬 博文
\vspace{0.2cm}\\ 
%%% 主任指導教員名
\normalsize{主任指導教員: 鎌谷 研吾}

%%% 主任指導教員以外の共同研究者については,本文中に明記して下さい 
}

\end{center}


\section{モンテカルロ法の現代史}

Metropolis法は戦時中にロスアラモスで導入されたENIACを見て着想されたアルゴリズムであり,戦争がさらに長引いていればENIACを用いた中性子拡散のシミュレーションに利用される予定であった.戦後に Metropolis and Ulam (1949) で世界に公開されたのち,実に半世紀もの間,物理学・統計学で使われていたほとんど\.{全}\.{て}のモンテカルロ法の指導原理となっていた.

このMetropolis法が高次元に弱いことはかねてから指摘されていたが,Chung et. al. (1987), Diaconis et. al. (2000) などの地道な発展を通じ,21世紀に入るとアルゴリズムの「対称性」を破ることで収束を速めることができることが理解されるようになった.
例えば有限空間$\{1,\cdots,n\}$上のランダムウォーク$X_{n+1}=X_n+\epsilon_{n+1}$を簡単に
\[X_{n+1}=2X_n+\epsilon_{n+1},\qquad\epsilon_{n+1}\sim\operatorname{Unif}(\{0,\pm1\})\]
と非対称化しただけでも,収束までの時間は$O(n^2)$から$O(n)$に改善する(が,このようなアプローチではこれ以上の改善ができないことも示せる).Diaconis (2013) が詳しい.

この考え方をさらに洗練させた手法群が,2010年代に(またしても)計算物理の分野で次々と提案された.最も有名なものは Bernard et. al. (2009) と Turitsyn et. al. (2011) である.

Faulkner et. al. (2018) でこの手法が世界初の氷の液相転移の全原子シミュレーションに成功するまで改良されると,統計学にも\textbf{区分確定的モンテカルロ法} (Piecewise Deterministic Monte Carlo,以下PDMC) として輸入された.特に代表的なものは Bouchard-Côté et. al. (2018) による Bouncy Particle サンプラーと Bierkens et. al. (2019) による Zig-Zag サンプラーである.

\section{区分確定的モンテカルロの夢:スケーラビリティ}

Metropolis法に基礎を持つ古典的なMCMCは,尤度の評価が必須でデータ数$N$に関し$O(N)$で計算複雑性が増える.
このままではビッグデータ時代の荒波には乗れず,時代の底に沈殿すべきアルゴリズムに間違いない.
しかしPDMCは対数尤度の勾配の不偏推定量があれば良い.
この性質により,SGDのようにデータの一部だけを見ながらアルゴリズムを進めることができ,
事後分布の最大値点さえ判ればデータ数に対して$O(1)$の効率が達成できる Bierkens et. al. (2019).

\section{現在の課題}

\subsection{高次元での効率}

有限空間の場合と違って,従来のMCMCよりもPDMCの方が本当に速くなっていることを示すことが理論的には難しく,アルゴリズムの根本的な違いから実証的に比較することも難しい.
前述のBouncy ParticleとZig-Zagは極めて単純なアルゴリズムを持ち,
例えばMichel et. al. (2020)によるForward ECMCなどの方が絶対に速いとみんなわかっているが,それが示せない.
% 少なくともHamiltonian Monte Carloと同じくらい速いとは思っている.

\subsection{アルゴリズム}

HMCには\texttt{Stan}があり,(確率的プログラミング言語としてのインターフェース,または手動で定義した)密度関数を与えるだけで,自動微分と適応的ハイパラチューニングを内部で実行して\texttt{C++}での高速なサンプリングを享受できる.
同様のことがPDMCでも可能であることはAndral and Kamatani (2024)が初めて指摘したことであり,それまではモデル毎に手動でアルゴリズムを組む必要があった.
これを受けてAndralは\texttt{pdmp\_jax}を,発表者は\texttt{PDMPFlux.jl}を開発した.

\begin{center}
\textbf{\large{参考文献}}
\end{center}
Metropolis, N. and Ulam, S. (1949). The Monte Carlo Method, \textit{Journal of the American Statistical Association}, \textbf{44}(247), 335-341.\\
Chung, F. R. K., Diaconis, P., and Graham, R. L. (1987). Random Walks Arising in Random Number Generation, \textit{The Annals of Probability}, \textbf{15}(3), 1148-1165.\\
Diaconis, P., Holmes, S., and Neal, R. M. (2000). Analysis of a Nonreversible Markov Chain Sampler, \textit{The Annals of Applied Probability}, \textbf{10}(3), 726-752.\\
Diaconis, P. (2013). Some Things We’ve Learned (about Markov Chain Monte Carlo), \textit{Bernoulli}, \textbf{19}(4), 1294-1305.\\
Bernard, E. P., Krauth, W., and Wilson, D. B. (2009). Event-chain Monte Carlo Algorithms for Hard-sphere Systems, \textit{Physical Review E}, \textbf{80}(5), 056704.\\
Turitsyn, K. S., Chertkov, M., and Vucelja, M. (2011). Irreversible Monte Carlo Algorithms for Efficient Sampling, \textit{Physica D: Nonlinear Phenomena}, \textbf{240}(4-5), 410-414.\\
Faulkner, M. F., Qin, L., Maggs, A. C., and Krauth, W. (2018). All-Atom Computations with Irreversible Markov Chains, \textit{The Journal of Chemical Physics}, \textbf{149}(6), 064113.\\
Bouchard-Côté, A., Vollmer, S. J., and Doucet, A. (2018). The Bouncy Particle Sampler: A Nonreversible Rejection-Free Markov Chain Monte Carlo Method, \textit{Journal of the American Statistical Association}, \textbf{113}(522), 855-867.\\
Bierkens, J., Fearnhead, P., and Roberts, G. (2019). The Zig-Zag Process and Super-Efficient Sampling for Bayesian Analysis for Big Data, \textit{The Annals of Statistics}, \textbf{47}(3), 1288-1320.\\
Bierkens, J., Kamatani, K., and Roberts, G. (2022). High-Dimensional Scaling Limits of Piecewise Deterministic Sampling Algorithms, \textit{The Annals of Applied Probability}, \textbf{32}(5), 3361-3407.\\
Michel, M., Durmus, A., and Sénécal, S. (2020). Forward Event-Chain Monte Carlo: Fast Sampling by Randomness Control in Irreversible Markov Chains, \textit{Journal of Comuptational and Graphical Statistics}, \textbf{29}(4), 689-702.\\
Andral, C. and Kamatani, K. (2024). Automated Techniques for Efficient Sampling of Piecewise-Deterministic Markov Processes, \textit{arXiv}: 2408.03682.

\end{document}
