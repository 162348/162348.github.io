% Options for packages loaded elsewhere
\PassOptionsToPackage{unicode}{hyperref}
\PassOptionsToPackage{hyphens}{url}
\PassOptionsToPackage{dvipsnames,svgnames,x11names}{xcolor}
%
\documentclass[
  11pt,
]{article}

\usepackage{amsmath,amssymb}
\usepackage{iftex}
\ifPDFTeX
  \usepackage[T1]{fontenc}
  \usepackage[utf8]{inputenc}
  \usepackage{textcomp} % provide euro and other symbols
\else % if luatex or xetex
  \usepackage{unicode-math}
  \defaultfontfeatures{Scale=MatchLowercase}
  \defaultfontfeatures[\rmfamily]{Ligatures=TeX,Scale=1}
\fi
\usepackage[sfmath]{kpfonts}
\ifPDFTeX\else  
    % xetex/luatex font selection
\fi
% Use upquote if available, for straight quotes in verbatim environments
\IfFileExists{upquote.sty}{\usepackage{upquote}}{}
\IfFileExists{microtype.sty}{% use microtype if available
  \usepackage[]{microtype}
  \UseMicrotypeSet[protrusion]{basicmath} % disable protrusion for tt fonts
}{}
\makeatletter
\@ifundefined{KOMAClassName}{% if non-KOMA class
  \IfFileExists{parskip.sty}{%
    \usepackage{parskip}
  }{% else
    \setlength{\parindent}{0pt}
    \setlength{\parskip}{6pt plus 2pt minus 1pt}}
}{% if KOMA class
  \KOMAoptions{parskip=half}}
\makeatother
\usepackage{xcolor}
\setlength{\emergencystretch}{3em} % prevent overfull lines
\setcounter{secnumdepth}{-\maxdimen} % remove section numbering


\providecommand{\tightlist}{%
  \setlength{\itemsep}{0pt}\setlength{\parskip}{0pt}}\usepackage{longtable,booktabs,array}
\usepackage{calc} % for calculating minipage widths
% Correct order of tables after \paragraph or \subparagraph
\usepackage{etoolbox}
\makeatletter
\patchcmd\longtable{\par}{\if@noskipsec\mbox{}\fi\par}{}{}
\makeatother
% Allow footnotes in longtable head/foot
\IfFileExists{footnotehyper.sty}{\usepackage{footnotehyper}}{\usepackage{footnote}}
\makesavenoteenv{longtable}
\usepackage{graphicx}
\makeatletter
\def\maxwidth{\ifdim\Gin@nat@width>\linewidth\linewidth\else\Gin@nat@width\fi}
\def\maxheight{\ifdim\Gin@nat@height>\textheight\textheight\else\Gin@nat@height\fi}
\makeatother
% Scale images if necessary, so that they will not overflow the page
% margins by default, and it is still possible to overwrite the defaults
% using explicit options in \includegraphics[width, height, ...]{}
\setkeys{Gin}{width=\maxwidth,height=\maxheight,keepaspectratio}
% Set default figure placement to htbp
\makeatletter
\def\fps@figure{htbp}
\makeatother

\makeatletter
\@ifpackageloaded{caption}{}{\usepackage{caption}}
\AtBeginDocument{%
\ifdefined\contentsname
  \renewcommand*\contentsname{Table of contents}
\else
  \newcommand\contentsname{Table of contents}
\fi
\ifdefined\listfigurename
  \renewcommand*\listfigurename{List of Figures}
\else
  \newcommand\listfigurename{List of Figures}
\fi
\ifdefined\listtablename
  \renewcommand*\listtablename{List of Tables}
\else
  \newcommand\listtablename{List of Tables}
\fi
\ifdefined\figurename
  \renewcommand*\figurename{図}
\else
  \newcommand\figurename{図}
\fi
\ifdefined\tablename
  \renewcommand*\tablename{Table}
\else
  \newcommand\tablename{Table}
\fi
}
\@ifpackageloaded{float}{}{\usepackage{float}}
\floatstyle{ruled}
\@ifundefined{c@chapter}{\newfloat{codelisting}{h}{lop}}{\newfloat{codelisting}{h}{lop}[chapter]}
\floatname{codelisting}{Listing}
\newcommand*\listoflistings{\listof{codelisting}{List of Listings}}
\makeatother
\makeatletter
\makeatother
\makeatletter
\@ifpackageloaded{caption}{}{\usepackage{caption}}
\@ifpackageloaded{subcaption}{}{\usepackage{subcaption}}
\makeatother
\makeatletter
\@ifpackageloaded{fontawesome5}{}{\usepackage{fontawesome5}}
\makeatother
\ifLuaTeX
  \usepackage{selnolig}  % disable illegal ligatures
\fi
\usepackage{bookmark}

\IfFileExists{xurl.sty}{\usepackage{xurl}}{} % add URL line breaks if available
\urlstyle{same} % disable monospaced font for URLs
\hypersetup{
  colorlinks=true,
  linkcolor={blue},
  filecolor={Maroon},
  citecolor={Blue},
  urlcolor={minty},
  pdfcreator={LaTeX via pandoc}}

\usepackage[top=15truemm,bottom=15truemm,left=10truemm,right=10truemm]{geometry}
\usepackage{titlesec}
\usepackage{picture}
\usepackage{fontawesome5}
\usepackage{luatexja}
\definecolor{minty}{HTML}{80c4ac}

\titleformat{\section}[block]
{}{}{0pt}
{
    \colorbox{minty}{\begin{picture}(0,10)\end{picture}}
    \hspace{0pt}
    \normalfont \Large\bfseries
    \hspace{-4pt}
}
[
\begin{picture}(100,0)
    \put(3,18){\color{minty}\line(1,0){300}}
\end{picture}
\\
\vspace{-30pt}
]

\titlespacing{\section}{0pc}{3.5ex plus .1ex minus .2ex}{1.5ex minus .1ex}

\renewcommand{\labelitemi}{\textcolor{minty}{\faCheckCircle}} %https://mirrors.ibiblio.org/CTAN/fonts/fontawesome/doc/fontawesome.pdf
% \faPaperclip が文献の列挙に良いかも.\author{}
\date{}

\begin{document}

\begin{figure}

\begin{minipage}{0.50\linewidth}
\Huge 司馬博文(しばひろふみ)\end{minipage}%
%
\begin{minipage}{0.50\linewidth}

\color{minty}

\hfill {\faIcon{home}} \url{https://162348.github.io/}

\par

\hfill {\faIcon{envelope}}
\href{mailto:shiba.hirofumi@ism.ac.jp}{\nolinkurl{shiba.hirofumi@ism.ac.jp}}

\par

\end{minipage}%

\end{figure}%

\vspace{-1em}

\section{8/28/2024 現在}\label{ux73feux5728}

総合研究大学院大学(統計科学コース)5年一貫博士課程2年目.

日本語,中国語が母語で,英語も話せる(TOEFL iBT 100点).Python, R
でのコーディング経験3年以上,Julia 1年以上.

\section{研究分野}\label{ux7814ux7a76ux5206ux91ce}

\begin{itemize}
\item
  輸送によるサンプリング法

  特に,シュレーディンガー橋や正規化フローなどの生成モデリング手法.
\item
  モンテカルロ法

  特に,逐次モンテカルロ法(SMC)やマルコフ連鎖モンテカルロ法(MCMC)などのベイズ統計計算アルゴリズム.
\item
  統計モデリング

  特に,政治学,疫学,惑星地球科学などの分野への応用.
\item
  ベイズ機械学習

  特に,ガウス過程や階層モデリング,ノンパラメトリクスなど.
\item
  データ駆動科学

  特に,データ埋め込み,可視化,軌道推定,データ同化など.
\end{itemize}

\section{学歴}\label{ux5b66ux6b74}

\renewcommand{\labelitemi}{\textcolor{minty}{\faGraduationCap}}

\begin{itemize}
\item
  \textbf{博士(統計科学)}.
  \emph{総合研究大学院大学先端学術院統計科学コース}. \hfill {2023.4 --
  2028.3}

  指導教員:\href{https://sites.google.com/view/kengokamatani/home}{鎌谷研吾教授},\href{https://sites.google.com/site/kyanostat/}{矢野恵佑准教授}
\item
  \textbf{学士(理学)}. \emph{東京大学理学部数学科}. \hfill {2019.4 --
  2023.3}

  指導教員:\href{https://www.ms.u-tokyo.ac.jp/~nakahiro/hp-naka-e}{吉田朋広教授}
\end{itemize}

\section{職歴}\label{ux8077ux6b74}

\renewcommand{\labelitemi}{\textcolor{minty}{\faUniversity}}

\begin{itemize}
\item
  \textbf{リサーチ・アシスタント}. \emph{統計数理研究所}. \hfill {2023.7
  -- 現在}

  確率過程の統計推測のための R パッケージである
  \href{https://r-forge.r-project.org/projects/yuima/}{YUIMA}
  の開発などを通じ,モンテカルロ法とベイズ統計の応用に取り組む.
\item
  \textbf{連携研究員}. \emph{東京大学先端科学技術研究センター}.
  \hfill {2023.4 -- 現在}

  信頼できる AI と機械学習の視点からの経済安全保障の研究.
\item
  \textbf{データサイエンティスト}. \emph{IMIS 研究所}. \hfill {2022.8 --
  2024.1}

  製造業のクライアントに対して統計分析と機械学習のソリューションを提供.
\end{itemize}

\section{研究滞在}\label{ux7814ux7a76ux6edeux5728}

\begin{itemize}
\item
  \textbf{ユニバーシティ・カレッジ・ロンドン},イギリス.\hfill {2024.11.4
  -- 2024.12.2}

  受入教員:Alexandros Beskos 教授
\end{itemize}

\renewcommand{\labelitemi}{\textcolor{minty}{\faBookmark}}



\end{document}
